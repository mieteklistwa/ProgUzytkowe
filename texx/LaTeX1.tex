\documentclass{article}
\usepackage[a4paper,left=3.5cm,right=2.5cm,top=2.5cm,bottom=2.5cm]{geometry}
%%\usepackage[MeX]{polski}
%%\usepackage[cp1250]{inputenc}
\usepackage{polski}
\usepackage[utf8]{inputenc}
\usepackage[pdftex]{hyperref}
\usepackage{makeidx}
\usepackage[tableposition=top]{caption}
\usepackage{algorithmic}
\usepackage{graphicx}
\usepackage{enumerate}
\usepackage{multirow}
\usepackage{amsmath} %pakiet matematyczny
\usepackage{amssymb} %pakiet dodatkowych symboli
\usepackage[tabele]{xcolor}
\usepackage{booktabs}
\usepackage{sidecap}
\usepackage{wrapfig}
\usepackage{subcaption}
\begin{document}

\title{Miasta Bonaire}

\maketitle

Sześciokątna gwiazda na fladze Bonaire reprezentuje sześć oryginalnych miejscowości, które zostały założone na wyspie. Przez lata ze wzrostem populacji oraz urbanizacji pięć z nich zostały przyłączone do stolicy Kralendijk. Tylko Rincon, który znajduje się w północnej części wyspy, pozostał oddzielnym miastem. Oprócz tych miast, powstały wiele nowych osad.

Według danych oficjalnych pochodzących z 2007 roku Bonaire (holenderskie terytorium zamorskie) posiadał 6 miejscowości o ludności przekraczającej 1 tys. mieszkańców. Stolica kraju Kralendijk jako jedyne miasto liczyło ponad 10 tys. mieszkańców; reszta miejscowości poniżej 5 tys. mieszkańców.




\begin{figure}[h!]
 \caption{Bonaire}
\centering
  \includegraphics[width=0.5\textwidth]{bonaire.png}

\end{figure}

\begin{table}[here]
\centering \caption{Największe miejscowości na Bonaire według liczebności mieszkańców (stan na 25.04.2007):}
\begin{tabular}{|r|r|r|r|}
  \hline
  $L.p.$ & $Miasto$ & $Dzielnica$ & $Liczba ludności$ \\
  \hline
  $1.$ & $Kralenijk$ & $-$ & $12.531$ \\
  \hline
  $ $ & $ $ & $Antriol$ & $3 947$ \\
  \hline
  $ $ & $ $ & $Nikiboko$ & $2 633$ \\
  \hline
  $ $ & $ $ & $Nort Salinja$ & $2 378$ \\
  \hline
  $ $ & $ $ & $Playa (centrum Kralendijk)$ & $1963$ \\
  \hline
  $ $ & $ $ & $Tera Kora$ & $1 610$ \\
  \hline
  $2.$ & $Rincon$ & $-$ & $1 788$ \\
  \hline
\end{tabular}
\end{table}



\end{document}

